\documentclass{article}

\usepackage{hyperref}
\usepackage{url}

\renewcommand\refname{}
\renewcommand\contentsname{Indice}

\title{Progettazione e Sviluppo di una applicazione mobile di Marketing ed E-Commerce}
\date{2016-11}
\author{Tommaso Berlose}

\begin{document}
  \pagenumbering{gobble}
  \maketitle
  \newpage
  \pagenumbering{arabic}
  \tableofcontents
  \newpage
  
  \section*{Intro}
  \newpage
  \section{Situazione esistente}

  \subsection{App Economy}

  \subsection{Ecosistema MyGelato}

  \subsection{Stato dell’Arte dell’Applicazione Android}
  \newpage
  \section{Specifiche di Progetto}

  \subsection{Rifare app uguale}

  \subsection{temi con flavors}

  \subsection{risorse da adattare}

  \subsection{parallelismo con iOS}

  \subsection{multilingua, login facebook, ecc...}
  \newpage
  \section{Scelte di Progetto}
  \paragraph{}
Definiti gli obiettivi progettuali dell'applicazione si sono valutati gli strumenti di sviluppo da utilizzare durante lo svolgimento della tesi: come parametri si é tenuto conto di tempistiche di aggiornamento, adeguamento alle linee guida del sistema operativo in oggetto, documentazione disponibile e modernità delle tecnlogie utilizzate.

Si sono inoltre adottate alcune strategie implementative, come la creazione di custom view, per poter meglio strutturare il progetto favorendone anche, in futuro, eventuali modifiche o ampliamenti.

\subsection{Strumenti di Sviluppo}

Poichè l'applicazione mobile, oggetto di tesi, era da sviluppare in ambito Android, si è scelto di utilizzare come principale strumento di sviluppo Android Studio.
In questo modo si è potuto programmare nativamente limitando la portabilità del codice su altre piattaforme ma sfruttando pienamente l'architettura del sistema operativo sottostante.
Come sistema di versioning si è utilizzato 


\subsubsection{Android Studio}

Android Studio \cite{HTMLIT:1} è un ambiente di sviluppo integrato (IDE) per lo sviluppo per la piattaforma Android. È stato annunciato il 16 maggio 2013 in occasione della conferenza Google I/O e la prima build stabile fu rilasciata nel dicembre del 2014.
Basato sul software della JetBrains IntelliJ IDEA, Android Studio è stato progettato specificamente per lo sviluppo di Android.[4] È disponibile il download su Windows, Mac OS X e Linux,[5][6] e sostituisce gli Android Development Tools (ADT) di Eclipse, diventando l' IDE primario di Google per lo sviluppo nativo di applicazioni Android.
Permette di creare un progetto gradle, apk, github, 

\subsection{Programmazione Nativa}
Nativo sì, nativo no. Qual è l’approccio migliore? Sicuramente entrambi hanno i loro pro e contro. Mentre da un lato il nativo offre la possibilità di una gestione totale del dispositivo senza la paura di trovare limiti, d’altra parte richiede spesso una programmazione molto professionale e si concentra esclusivamente su una piattaforma impedendo un’agile riciclo dei propri sforzi su altri mercati del mobile.

Il non-nativo – anche se è impossibile generalizzare data la diversità degli ambienti appena citati – offre vantaggi vari, ascrivibili a volte ad una minore necessità di programmare e molto spesso alla possibilità di creare applicazioni cross-platform distribuibili su sistemi operativi diversi.

\subsubsection{Java + XML}

\subsection{Database}

\subsubsection{Realm}

\subsection{Chiamate Server}

\subsubsection{OkHttp}

\subsubsection{API REST}

\subsection{Gestore di Eventi}

\subsubsection{EventBus}

\subsection{Notifiche Push}

\subsubsection{Geofencing}

\subsubsection{Firebase}

\subsection{E-Commerce}

\subsubsection{Stripe}

\subsection{Librerie Minori}

\subsection{Scelte progettuali strane e che non so come chiamare}

\subsubsection{Flavors’ Custom View}

  \newpage
\section{Implementazione}

\subsection{Shop}

\subsubsection{Ricerca}

\subsubsection{Preferiti}

\subsection{Carte gelato}

\subsubsection{Gelato Master}

\subsubsection{Notifiche Push}

\subsection{Coupon}

\subsubsection{Acquisto}

\subsubsection{Regalo}

\subsubsection{Utilizzo e Validazione}

  \newpage
\section{Performance e Supporto}

\subsection{Profiling}

\subsection{Supporto vecchi device}

  \newpage
\section{Conclusioni}

\newpage
\section{Bibliografia}
\bibliographystyle{ieeetr}
\bibliography{bibliografia}

  \newpage
\section{Figure}

  \newpage
\section{Codice}

  \newpage
\section{Ringraziamenti}


\end{document}