\section{Specifiche Progettuali}

L'ecosistema MyGelato è un progetto ampio che si basa sulla cooperazione tra più elementi fondamentali: un’applicazione basata sul framework Ruby on Rails che implementa le funzionalità di creazione, gestione e fruizione di contenuti multimediali affiancate da un sistema e-commerce per la compravendita di coupon digitali.
Per poter usufruire di tali servizi è necessario l'utilizzo di un'applicazione mobile, chiamata MyGelato, disponibile per iOS e Android; quest'ultima argomento di questa tesi.

Essendo l'applicazione Android solo una compenente di un più ampio progetto è essenziale fin dalla prima fase di sviluppo valutare correttamente e nella loro completezza tutte le specifiche date dall'azienda che ha commissionato il lavoro e date dalla necessità di far cooperare l'applicazione con altri componenti.
Si valutano inizialmente le specifiche strutturali e logiche che determinano le interazioni principali dell'utente con l'applicativo: navigazione, design e flussi logici.
Questi macro argomenti permeano ogni singolo componente del sistema e per questo devono essere valutati prima di iniziare qualsiasi tipo di sviluppo poichè serve considerare le interazioni tra i flussi logici e il design da dover mantenere coerente all'interno di tutta interfaccia utente.

Prima di tutto si studia come implementare in maniera semplice e intuitiva il design richiesto che comprende la possibilità di modificare il tema principale dell'applicazione in base a \textit{gusti} differenti.
Sono state consegnate insieme alle specifiche anche le risorse da utilizzare per personalizzare ogni singolo elemento grafico dell'interfaccia in modo da diversificarla rispetto al design suggerito da Google e che standardizza la maggior parte delle app Android.

Fatte queste considerazioni si valuta come implementare una navigazione semplice ed intuiva per favorire l'utente nella scelta delle principali funzioni disponibili all'interno dell'applicazione: \textit{Acquisto/Utilizzo Coupon}, \textit{Lista Gelaterie Preferite}, \textit{Mappa per la Ricerca}, \textit{Mastro Gelatiere} e \textit{Cambio Tema}.
Ogni sezione fa parte dei due principali flussi logici presenti all'interno dell'applicativo: quello di marketing digitale che permette l'esplorazione da parte dell'utente delle gelaterie in una determinata zona e delle relative promozioni; e quello di e-commerce che permette l'acquisto online, il controllo, il regalo e l'utilizzo dei coupon gelato resi disponibili dalle gelaterie del circuito MyGelato.

Infine si valutano le specifiche tecnologiche richieste sia dall'azienda che dalla necessaria cooperazione con gli altri elementi del progetto.
Vi deve essere la possibilità di avere un'applicazione multilingua, con autenticazione tramite social network e compatibile con la maggior parte dei device android attualmente sul mercato.
Sono di forte impatto anche alcune scelte a livello di comunicazione con il backend che indirizza lo sviluppo in una comunicazione bidirezionale tra applicazione e backend tramite API REST e notifiche push.
Essenziale oltretutto mantenere la coerenza tra le due applicazioni mobile sviluppate per le piattaforme iOS e Android, rispettando in ognuna i propri pattern tipici; insieme al sistema gestionale online presente per gli amministratori del sistema.

\subsection{Design}


\subsection{Ricerca}
L'elemento centrale di tutto l'applicazione è la possibilità di effettuare una ricerca delle gelaterie, iscritte al circuito MyGelato, presenti in una determinata zona così da ottenere alcune informazioni essenziali sull'esercizio e le relative promozioni disponibili.
Questa funzionalità è, infatti, l'elemento cardine che lega ogni possibile azione dell'utente all'interno dell'applicazione: è essenziale sia per il sistema di marketing che per il sistema di e-commerce.

Ogni utente, anche se non registrato, deve poter accedere ad una mappa che visualizzi tutti gli shop disponibili, sia sotto forma di marker visibili in base al luogo di ricerca sia sotto forma di lista ordinata in base alla distanza dall'utente.
Ogni elemento, scaricato e aggiornato in base alle ultime informazioni presenti sul server, permette di ottenere il nome della gelateria, l'indirizzo ed un eventuale recapito telefonico.

Questo tipo di ricerca permette all'utente di esplorare la zona attorno al luogo in cui trova, o anche ad un luogo di suo interesse, così da ottenere informazioni, sia di carattere commerciale sia pubblicitario, riguardo agli esercizi presenti in maniera molto diretta; permettendo anche di salvare lo shop all'interno della lista preferiti.
Nel frattempo ogni gelateria presente all'interno del sistema ottiene la possibilità di essere trovata anche da utenti nuovi che non conoscono la zona e di pubblicizzarsi tramite una strategia di marketing unificata e standardizzata: le carte promozionali.

Oltre al sistema di marketing la ricerca delle gelaterie è essenziale per il sistema di e-commerce poichè la mappa viene utilizzata per la scelta da parte dell'utente della gelateria per la quale vuole acquistare un determinato coupon.
La ricerca è quindi presente sia nel sistema di navigazione principale dell'applicazione sia raggiungibile all'interno di ogni flusso logico scelto dall'utente.

Come si può vedere in figura l'abbozzo della mappa permette di capire la necessità di avere un'implementazione semplice e facilmente utilizzabile da qualsiasi utente, dove sia possibile passare dalla visualizzazione a mappa a quella a lista e dove siano già visibili le principali informazioni riguardo ad ogni gelateria.


FIGGGGGGGGGGGGGGGGGG MAPPA


\subsection{Marketing Digitale}
Uno dei due flussi logici principali presenti all'interno dell'applicazione riguarda il marketing digitale che viene svolto per le gelaterie legate all'ecosistema MyGelato.
Il sistema di marketing ha lo scopo di rendere più semplice la ricerca delle gelaterie vicino all'utente, sponsorizzarne eventuali promozioni e permettere all'utente di salvare gli esercizi preferiti così da rimanere aggiornato sulle nuove sponsorizzazioni.
Queste funzionalità, disponibili in buona parte per qualsiasi utente anche non registrato, seguono un flusso logico che parte dalla ricerca degli shop e trova effetto principale nella scoperta e nell'utilizzo delle carte promozionali, della singola gelateria e del Mastro Gelatiere.

\subsubsection{Carte Promozionali}
Le carte promozionali sono il principale mezzo di advertising all'interno dell'applicazione: sono dei volantini digitali formati da un'immagine, un titolo, un testo descrittivo, un eventuale recapito telefonico e un eventuale link per avere maggiori informazioni.
Ci sono due tipologia di carte: le carte specifiche di ogni esercizio che sono i volantini informativi della singola gelateria e le carte del mastro gelatiere che invece si rifanno alle promozioni generiche proposte per l'intero sistema.

$ DA QUI  $




Sono inoltre visualizzabili le Carte dello shop che altro non sono che le ultime promozioni legate all'attività: sconti, novità, messaggi promozionali.
Grazie a queste informazioni l'utente può ottenere oltre le informazioni genericamente trovabili online, anche delle informazioni pubblicitarie.
Dall'altra parte del sistema la gelaterie ha la possibilità di ottenere una maggior copertura pubblicitaria tramite l'utilizzo di un sistema centralizzato e slegato da sistemi di advertising.

\subsubsection{Preferiti}
Elemento molto importante, presente all'interno della navigazione principale dell'applicazione, è la gestione delle proprie gelaterie preferite.
Questo sistema permette ad ogni utente, registrato o meno, di salvare offline sul proprio dispositivo gli esercizi di maggiore interesse.

Questo dà modo all'utente di poter accedere in ogni momento, specialmente in mobilità, alle informazioni che più gli interessavano di ogni gelateria: indirizzo, recapito telefonico ed eventuali promozioni.
La ricerca in questo caso è molto semplicificata poichè viene presentata una sezione a parte con una lista degli shops preferiti, senza dover obbligare l'utente ad effettuare una nuova ricerca all'interno della mappa.

Il salvataggio all'interno dei preferiti avviene direttamente all'interno della ricerca, sia tramite la mappa che tramite la lista ordinata in base ala distanza dall'utente grazie ad un'icona esemplificativa.

Questa funzionalità è pensata principalmente per fidelizzare il consumatore: ogni utente avendo la possibilità di salvare una gelateria ha anche la possibilità di ottenere velocemente informazioni su nuove promozioni, ottenere velocemente i contatti dell'esercizio come se li avesse salvati in rubrica e valuatare ogni volta la distanza tra se e lo shop.

Il passo successivo in questo senso è dato dal rendere bidirezionale questo collegamento, rendendo in alcuni casi non necessaria la ricerca da parte dell'utente di nuove informazioni dandogliele invece ad ogni aggiornamento.
Per fare questo, solo nel caso di utenti registrati, l'aggiunta di uno shop ai preferiti include alcune funzionalità di maggiore visibilità: geofencing e notifiche push.
Il geofencing permette di ricevere una notifica ogni qualvolta l'utente si trovi a meno di 5 km da una delle proprie gelaterie, così da essere informato di essere vicino in termini di localizzazione.
Le notifiche push invece sono attivate per avvertire un utente che una delle proprie gelaterie ha pubblicato una nuova promozione tramite l'aggiunta una carta sul sistema MyGelato.
L'utente così rimane informato costantemente delle ultime promozioni disponibili e il propietario di una gelateria ha la certezza di effettuare pubblicità diretta tra se e i suoi clienti più affezzionati.


\subsection{E-Commerce}

\subsubsection{Acquisto Coupon}

\subsubsection{Utilizzo Coupon}

\newpage