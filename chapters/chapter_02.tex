\section{Specifiche Progettuali}

L'ecosistema MyGelato è un progetto ampio che si basa sulla cooperazione tra più elementi fondamentali: un’applicazione basata sul framework Ruby on Rails che implementa le funzionalità di creazione, gestione e fruizione di contenuti multimediali affiancate da un sistema e-commerce per la compravendita di coupon digitali.
Per poter usufruire di tali servizi è necessario l'utilizzo di un'applicazione mobile, chiamata MyGelato, disponibile per iOS e Android; quest'ultima argomento di questa tesi.

Essendo l'applicazione Android solo una compenente di un più ampio progetto è essenziale fin dalla prima fase di sviluppo valutare correttamente e nella loro completezza tutte le specifiche date dall'azienda che ha commissionato il lavoro e tutte quelle date dalla necessità di far cooperare l'applicazione con altri componenti.
Si valutano inizialmente le specifiche strutturali e logiche che determinano le interazioni principali dell'utente con l'applicativo: navigazione, design e flussi logici.
Questi macro argomenti permeano ogni singolo componente del sistema e per questo devono essere valutati prima di iniziare qualsiasi tipo di sviluppo: serve considerare le interazioni tra i flussi logici e valutare come mantenere coerente il design all'interno di tutta l'interfaccia utente.

Prima di tutto si studia come implementare in maniera semplice e intuitiva il design richiesto che comprende la possibilità di modificare il tema principale dell'applicazione in base a \textit{gusti} differenti.
Sono state consegnate insieme alle specifiche anche le risorse da utilizzare per personalizzare ogni singolo elemento grafico dell'interfaccia in modo da diversificarla rispetto al design suggerito da Google e che standardizza la maggior parte delle app Android.

Fatte queste considerazioni si valuta come implementare una navigazione semplice ed intuiva per favorire l'utente nella scelta delle principali funzioni disponibili all'interno dell'applicazione: \textit{Acquisto/Utilizzo Coupon}, \textit{Lista Gelaterie Preferite}, \textit{Mappa per la Ricerca}, \textit{Mastro Gelatiere} e \textit{Cambio Tema}.
Ogni sezione fa parte dei due principali flussi logici presenti all'interno dell'applicativo: quello di marketing digitale che permette l'esplorazione da parte dell'utente delle gelaterie in una determinata zona e delle relative promozioni; e quello di e-commerce che permette l'acquisto online, il controllo, il regalo e l'utilizzo dei coupon gelato resi disponibili dalle gelaterie del circuito MyGelato.

Si valutano poi le specifiche tecnologiche richieste sia dall'azienda che dalla necessaria cooperazione con gli altri elementi del progetto.
Vi deve essere la possibilità di avere un'applicazione multilingua, con autenticazione tramite social network e compatibile con la maggior parte dei device android attualmente sul mercato.
Sono di forte impatto anche alcune scelte a livello di comunicazione con il backend che indirizza lo sviluppo in una comunicazione bidirezionale tra applicazione e backend tramite API REST e notifiche push.
Essenziale infine mantenere la coerenza tra le due applicazioni mobile sviluppate per le piattaforme iOS e Android, rispettando in ognuna i propri pattern tipici, insieme al sistema gestionale online presente per gli amministratori del sistema.

\subsection{Design}
Il design dell'applicazione è uno degli elementi più particolari dell'applicazione: sviluppato fin nel dettaglio permette di personalizzare in maniera coerente anche i componenti più basilari dell'interfaccia utente.
Sono presenti tre font custom che devono essere utilizzati in situazioni e contesti differenti in modo che l'utente fin dal primo colpo d'occhio capisca quali siano i messaggi informativi e quali siano quelli di contorno.

Le risorse grafiche rese disponibili all'interno delle specifiche richieste servono a rendere coerente l'interfaccia utente in ogni suo elemento: fin dall'icona per chiudere un'Activity a concludere con ogni elemento della navigazione principale.
Il fatto di avere un'insieme di risorse ben organizzate è necessario per rendere la programmazione più semplice, così da includere di volta in volta in maniera dinamica immagini in base alla lingua dell'applicazione o in base al tema chiaro/scuro.

In figura è possibile vedere alcune delle immagini presentate per la navigazione principale, in tutte le loro versioni.

FIGGG

Altro elemento fondamentale presente in quasi ogni Activity dell'applicativo è il tema dell'applicazione.
I temi sono principalmente delle colorazioni differenti che devono caratterizzare e modificare il main color della UI cambiando anche i testi e le icone in base al fatto di essere temi chiari o scuri.
Ogni tema, che deve essere modificabile a piacimento dall'utente direttamente da una funzione raggiungibile dalla navigazione principale, ha come nome un gusto differente di gelato e un proprio set di risorse.

Sarà importante capire come implementare un sistema che implementi i temi in ogni componente della UI senza rendere lo sviluppo dell'applicazione in questo senso un collo di bottiglia.



\subsection{Ricerca}
L'elemento centrale di tutto l'applicazione è la possibilità di effettuare una ricerca delle gelaterie, iscritte al circuito MyGelato, presenti in una determinata zona così da ottenere alcune informazioni essenziali sull'esercizio e le relative promozioni disponibili.
Questa funzionalità è, infatti, l'elemento cardine che lega ogni possibile azione dell'utente all'interno dell'applicazione: è essenziale sia per il sistema di marketing che per il sistema di e-commerce.

Ogni utente, anche se non registrato, deve poter accedere ad una mappa che visualizzi tutti gli shop disponibili, sia sotto forma di marker visibili in base al luogo di ricerca sia sotto forma di lista ordinata in base alla distanza dall'utente.
Ogni elemento, scaricato e aggiornato in base alle ultime informazioni presenti sul server, permette di ottenere il nome della gelateria, l'indirizzo ed un eventuale recapito telefonico.

Questo tipo di ricerca permette all'utente di esplorare la zona attorno al luogo in cui trova, o anche ad un luogo di suo interesse, così da ottenere informazioni, sia di carattere commerciale sia pubblicitario, riguardo agli esercizi presenti in maniera molto diretta; permettendo anche di salvare lo shop all'interno della lista preferiti.
Nel frattempo ogni gelateria presente all'interno del sistema ottiene la possibilità di essere trovata anche da utenti nuovi che non conoscono la zona e di pubblicizzarsi tramite una strategia di marketing unificata e standardizzata: le carte promozionali.

Oltre al sistema di marketing la ricerca delle gelaterie è essenziale per il sistema di e-commerce poichè la mappa viene utilizzata per la scelta da parte dell'utente della gelateria per la quale vuole acquistare un determinato coupon.
La ricerca è quindi presente sia nel sistema di navigazione principale dell'applicazione sia raggiungibile all'interno di ogni flusso logico scelto dall'utente.

Come si può vedere in figura l'abbozzo della mappa permette di capire la necessità di avere un'implementazione semplice e facilmente utilizzabile da qualsiasi utente, dove sia possibile passare dalla visualizzazione a mappa a quella a lista e dove siano già visibili le principali informazioni riguardo ad ogni gelateria.


FIGGGGGGGGGGGGGGGGGG MAPPA


\subsection{Marketing Digitale}
Uno dei due flussi logici principali presenti all'interno dell'applicazione riguarda il marketing digitale che viene svolto per le gelaterie legate all'ecosistema MyGelato.
Ha lo scopo di rendere più semplice la ricerca delle gelaterie vicino all'utente, sponsorizzarne eventuali promozioni e permettere all'utente di salvare gli esercizi preferiti così da rimanere aggiornato sulle nuove promozioni disponibili.
Queste funzionalità, disponibili in buona parte per qualsiasi utente anche non registrato, seguono un flusso logico che parte dalla ricerca degli shop e trova effetto principale nella scoperta e nell'utilizzo delle carte promozionali, della singola gelateria e del Mastro Gelatiere.

\subsubsection{Carte Promozionali}
Le carte promozionali sono il principale mezzo di advertising all'interno dell'applicazione: sono dei volantini digitali formati da un'immagine/video, un titolo, un testo descrittivo, un eventuale recapito telefonico e un eventuale link per avere maggiori informazioni.
Ci sono due tipologia di carte: le carte specifiche di ogni esercizio che sono i volantini informativi della singola gelateria e le carte del mastro gelatiere che invece si rifanno alle promozioni generiche proposte per l'intero sistema.

Le Carte del singolo esercizio altro non sono che le ultime promozioni legate all'attività: sconti, novità, messaggi promozionali.
L'utente in questo modo può facilmente reperire le informazioni pubblicitarie di una determinata gelateria online e in mobilità; considerando che difficilmente in questo campo vi è già una diffusa pubblicizzazione sui social network o sui sistemi di advertising.
Le gelaterie quindi hanno la possibilità di ottenere una maggior copertura pubblicitaria tramite l'utilizzo di un sistema centralizzato e standardizzato.

Le carte di uno stesso esercizio devono essere raggruppate in modo che l'utente possa scorrerle e visualizzare tutte le promozioni o informazioni disponibili insieme.
L'activity per questa visualizzazione deve essere raggiungibile ogni volta che sono mostrate le informazioni della gelateria: sia internamente alla ricerca che internamente alla lista dei preferiti dell'utente.

Oltre alle singole gelaterie l'ecosistema MyGelato prevede anche le carte del Mastro Gelaterie che comprendono promozioni, informazioni e novità pubblicate genericamente dagli amministratori del sistema e che sono valide per tutti gli esercizi facenti parte del circuito.
Essendo un insieme di carte totalmente generiche e di fondamentale importanta per il sistema di advertising creato, questa sezione deve essere inserita all'interno della navigazione principale del sistema.
Inoltre, come avverrà poi con il sistema di gelaterie preferite, ogni utente registrato verrà notificato di eventuali nuove promozioni nel momento stesso in cui diventeranno disponibili.

\subsubsection{Preferiti}
La gestione delle proprie gelaterie preferite è uno degli presenti all'interno della navigazione principale dell'applicazione, accessibile da qualsiasi utente anche non registrato: permette di salvare offline sul proprio dispositivo gli esercizi di maggiore interesse.

Questo dà modo all'utente di poter accedere in ogni momento, specialmente in mobilità, alle informazioni che più gli interessano di ogni gelateria: indirizzo, recapito telefonico ed eventuali promozioni.
La ricerca in questo caso è molto semplicificata poichè viene presentata una sezione a parte con una lista degli shops preferiti, senza dover obbligare l'utente ad effettuare una nuova ricerca all'interno della mappa.

Il salvataggio all'interno dei preferiti avviene direttamente all'interno della ricerca, sia tramite la mappa che tramite la lista ordinata in base ala distanza dall'utente grazie ad un'icona esemplificativa.

Questa funzionalità è pensata principalmente per fidelizzare il consumatore: ogni utente avendo la possibilità di salvare una gelateria ha anche la possibilità di ottenere velocemente informazioni su nuove promozioni, ottenere velocemente i contatti dell'esercizio come se li avesse salvati in rubrica e valuatare ogni volta la distanza tra se e lo shop.

Il passo successivo in questo senso è dato dal rendere bidirezionale questo collegamento, rendendo in alcuni casi non necessaria la ricerca da parte dell'utente di nuove informazioni dandogliele invece ad ogni aggiornamento.
Per fare questo, solo nel caso di utenti registrati, l'aggiunta di uno shop ai preferiti include le funzionalità di geofencing e notifiche push, quest'ultima presente anche per le carte promozionali del mastro gelatiere.

Il geofencing permette di ricevere una notifica ogni qualvolta l'utente si trovi a meno di 5 km da una delle proprie gelaterie, così da essere informato di essere vicino in termini di localizzazione.
Le notifiche push invece sono attivate per avvertire un utente che una delle proprie gelaterie ha pubblicato una nuova promozione tramite l'aggiunta una carta sul sistema MyGelato.

L'utente così rimane informato costantemente delle ultime promozioni disponibili e il propietario di una gelateria ha la certezza di effettuare pubblicità diretta tra se e i suoi clienti più affezzionati.

\subsection{E-Commerce}
Il secondo flusso logico presente all'interno dell'applicazione riguarda l'acquisto e l'utilizzo di coupon gelato in completa mobilità e online.
Questo sistema già più che diffuso in tantissimi altri ambiti del reseller online è uno dei cardini su cui maggiormente si vuole puntare sia per funzionalità sia per innovazione.

I Coupon gelato sono buoni acquisto che si possono acquistare dirattemente tramite l'applicazione e permettono, a chiunque ne sia virtualmente in possesso, di utilizzarli nelle gelaterie che li hanno rilasciati tramite validazione.
L'acquisto dei buoni deve poter essere fatto online e in mobilità utilizzando le ultime tecnologie disponibili per i pagamenti online; essenziale mantenere un occhio di riguardo alla sicurezza di questo tipo di transizioni.

Tra le specifiche risulta essere presente anche il concetto di condivisione dei coupon tramite un sistema di share/redeem: un utente che ha acquistato un coupon ha la possibilità di condividerlo con un altro utente registrato che può riscuotere poi il buono. La condivisione deve avvenire tramite applicazioni e canali di comunicazioni facilmente utilizzabili all'interno di uno smartphone in modo da rendere l'interno procedimento il più facile ed intuitivo possibile.

Infine vi deve essere ovviamente la possibilità di validare un coupon una volta che si decide di utilizzarlo all'interno della gelateria che lo ha rilasciato online tramite il circuito MyGelato. Questo procedimento prevede due entità in gioco, colui che possiede il coupon e il geleatio che deve poterlo validare; in entrambi i casi le funzionalità da implementare dovranno essere presenti all'interno dell'applicazione in base al tipo di account utilizzato così da mantenere coerenza negli strumenti utilizzati per collegarsi alla piattaforma.

\subsubsection{Coupon}
I Coupon sono buoni per l'acquisto di un bene materiale, normalmente gelati, che si può ritirare in qualsiasi momento in sede all'esercizio che ha effettuato la vendita.
Hanno un nome ed un valore, il prezzo, scelti durante l'inserimento tramite il sistema amministrativo gestionale che può inserire anche informazioni aggiuntive come la scadenza, il circuito di vendita e la valuta.
Sono generici per tutto l'ecosistema e sono quindi disponibili per ogni gelateria che li rende disponibili anche se all'acquisto sarà necessario specificare l'esercizio nel quale si desidereranno poi utilizzare.

All'interno della navigazione principale dell'applicazione sarà necessario avere una sezione apposita per poter gestire tutti i propri coupon: una lista di quelli utilizzati, quelli regalati ad altri utenti e quelli che si ponno ancora utilizzare personalmente.
Questa sezione sarà disponibile offline ma dovrà ogni volta essere sincronizzata con il server in modo da avere coerenza anche utilizzando la stessa applicazione con lo stesso account da dispositivi differenti.

Oltre alla possibilità di acquistare coupon si dovrà poter accedere direttamente alle funzioni di condivisione, riscatto e utilizzo così da avere una gestione centralizzata di tutto il flusso logico legato all'e-commerce.

\subsubsection{Acquisto}
L'acquisto di un coupon è permesso a qualsiasi utente che si sia registrato, tramite mail o social network, al circuito MyGelato.
Questa funzionalità è raggiungibile direttamente tramite parte della navigazione principale dell'applicazione, anche se deve essere disponibile solo nel caso in cui l'utente abbia eseguito il login, altrimenti dovranno essere richieste le credenziali di accesso.

Le operazioni richieste per l'acquisto di un coupon dovranno tutte essere racchiuse all'interno di una stessa activity in cui l'utente verrà guidato attraverso le varie fasi fino all'avvenuta transizione.
Il primo passo da svolgere è la scelta della gelateria su cui eseguire l'acquisto, come spiegato precedentemente si andrà a sfruttare anche in questo caso la mappa e la lista per la ricerca degli shop già utilizzata all'interno di altre funzionalità.
In questo caso però verranno visualizzati solo le gelaterie che permettono l'acquisto di coupon e quando verranno selezionati i marker non si otterenno tutte le informazioni sull'esercizio ma sarà presente un icona per selezionare lo shop.

Una volta selezionata la gelateria dovranno essere scaricati dal server i coupon disponibili e nel frattempo verrà resa disponibile la scelta del metodo di pagamento che si appoggerà al sistema di pagamento online Stripe.
Si potrà scegliere tramite una lista di metodi di pagamento quello che si preferirà usare che sarà successivamente impostato come predefinito.

Scelto il metodo di pagamento, scorrendo la lista dei coupon disponibili ed effettuando la scelta si potrà infine scegliere di completare l'acquisto che in caso di successo dovrà poi riportare l'utente sulla sezione di gestione dei coupon e altrimenti presntare un messaggio di errore.

\subsubsection{Condivisione e Riscatto}

\subsubsection{Utilizzo e Validazione}
































\newpage