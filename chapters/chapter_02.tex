\section{Specifiche Progettuali}
L'ecosistema MyGelato è un progetto ampio che si basa sulla cooperazione tra più elementi fondamentali - applicazione mobile, piattaforma di backend, sistema di comunicazione, sistema di e-commerce, ecc... - la maggior parte dei quali sviluppati da persone diverse con tempistiche differenti.
Durante l'iniziale studio di questo elaborato è stato quindi necessario tenere conto di determinate specifiche così da ottenere uno sviluppo lineare e il più semplice possibile.
Alle specifiche tecniche richieste dall'azienda durante la progettazione dell'ecosistema MyGelato, si sono quindi aggiunte alcune limitazioni legate alle scelte progettuali effettuate da altre persone durante lo sviluppo del loro applicativo.

Le scelte legate alle richieste da parte dell'azienda che ha commissionato il progetto riguardano principalmente quindi le scelte strutturali dell'applicativo come la navigazione, il design e i flussi logici presenti.
Aggiungendo alcune richieste tecnologiche come la possibilità di avere un'applicazione multilingua, con autenticazione tramite i social network e compatibile con la maggior parte dei device android attualmente sul mercato.

Invece le specifiche dettate dalla cooperazione tra le varie componenti del sistema hanno avuto un impatto molto maggiore sulle tecnologie scelte e utilizzate per lo sviluppo.
In particolar modo sono state di forte impatto alcune scelte a livello di comunicazione con il backend che ha indirizzato lo sviluppo in una comunicazione bidirezionale tra applicazione e backend tramite API REST e notifiche push.
Essenziale anche mantenere la coerenza tra le due applicazioni mobile sviluppate per le piattaforme iOS e Android; come anche per la piattaforma gestionale presente per gli amministratori del sistema.

Le specifiche progettuali di questo taglio prettamente tecnologico saranno trattate nel dettaglio all'interno delle scelte progettuali mentre quelle legate ai flussi logici dell'applicazione sono descritte nel dettaglio nei seguenti capitoli.

\subsection{Marketing Digitale}
Uno dei due flussi logici principali presenti all'interno dell'applicazione riguarda il marketing digitale che viene svolto per le gelaterie legate all'ecosistema MyGelato.
Il sistema di marketing ha lo scopo di rendere più semplice la ricerca delle gelaterie vicino all'utente, sponsorizzarne eventuali promozioni e permettere all'utente di salvare gli esercizi preferiti così da rimanere aggiornato su eventuali nuove sponsorizzazioni.

\subsubsection{Ricerca Shops}
L'intero sistema si basa sulla presenza di una mappa che visualizza tutti gli shop facente parte del sistema sia sotto forma di marker visibili in base al luogo di ricerca sia sotto forma di lista ordinata in base alla distanza dall'utente.
Ogni elemento, scaricato e aggiornato in base alle ultime informazioni presenti sul server, permette di aggiornarsi su nome della gelateria, indirizzo ed eventuale recapito telefonico.
Sono inoltre visualizzabili le Carte dello shop che altro non sono che le ultime promozioni legate all'attività: sconti, novità, messaggi promozionali.
Grazie a queste informazioni l'utente può ottenere oltre le informazioni genericamente trovabili online, anche delle informazioni pubblicitarie.
Dall'altra parte del sistema la gelaterie ha la possibilità di ottenere una maggior copertura pubblicitaria tramite l'utilizzo di un sistema centralizzato e slegato da sistemi di advertising.

\subsubsection{Shops Preferiti}
Dalla parte dell'utente vi è la possibilità di salvare ogni gelateria tra i preferiti, che verranno mantenuti anche offline insieme a tutte le informazioni legate alle attività scelte.
Questo permette di avere una sezione apposita da poter consultare per poter trovare le proprie gelaterie così da non dover rieffettuare ogni volta le ricerca e poter ottenere le informazioni che si desiderano più velocemente.

L'aggiunta di uno shop ai preferiti include inoltre alcune funzionalità comportando una maggiore visibilità: geofencing e notifiche push.
Il geofencing permette di ricevere una notifica ogni qualvolta l'utente si trovi a meno di 5 km da una delle proprie gelaterie, così da essere informato di essere vicino in termini di localizzazione.

Le notifiche push invece sono attivate per avvertire un utente che una delle proprie gelaterie ha pubblicato una nuova promozione tramite l'aggiunta una carta sul sistema MyGelato.
L'utente così rimane informato costantemente delle ultime promozioni disponibili me il propietario di una gelateria ha la certezza di avere un minimo di pubblicità diretta tra se e i suoi clienti più affezzionati.

\subsection{E-Commerce}

\subsubsection{Acquisto Coupons}

\subsubsection{Utilizzo Coupons}

\subsection{Design}

\newpage