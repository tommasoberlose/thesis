
\chapter{Progettazione}

Definiti gli obiettivi progettuali dell'applicazione si sono valutati
gli strumenti di sviluppo da utilizzare durante lo svolgimento della
tesi: come parametri si é tenuto conto di tempistiche di aggiornamento,
adeguamento alle linee guida del sistema operativo in oggetto, documentazione
disponibile e modernità delle tecnlogie utilizzate.

Durante la prima fase di sviluppo strutturale si sono valutate le
tecnologie da utilizzare in modo da tenere il passo con le ultime
specifiche di Android e tenendo conto che, anche secondo gli ultimi
report pubblicati da Google, rimane una forte frammentazione della
distribuzione del sistema operativo, dovuta ai molteplici vendor.
\autocite{ANDROIDSTUDIO:DASHBOARD} È quindi da tenere in considerazione
la retrocompatibilità delle librerie utilizzate, avendo scelto di
supportare fino alla versione 16 del SDK di Android (versione 4.1
Jelly Bean), e la possibilità di utilizzare l'applicazione anche su
device con schermi e hardware differenti.

Considerato infine che alcuni nodi centrali dell'elaborato necessitano
di specifiche minime in termini di affidabilità e sicurezza si è scelto
di utilizzare librerie esterne anche per le funzioni di utilizzo del
database e di connessione al server, rispettivamente Realm e OkHttp,
che oltre a semplificare lo sviluppo garantiscono un'alta gestione
degli errori.

\section{Sviluppo Android}

Questa tesi si inserisce all'interno del lavoro di creazione della
piattaforma MyGelato, il quale include il supporto sia alla piattaforma
Android sia alla piattaforma iOS. Essendosi però creati due differenti
progetti durante la prima fase di progettazione, si è potuto scegliere
come metodo implementativo l'utilizzo della programmazione nativa
Android che si basa su Java per la programmazione e su XML per la
creazione delle risorse.

Sicuramente lavorare in questo modo ha i propri pro e contro: mentre
da un lato il nativo offre la possibilità di una gestione totale del
dispositivo senza la paura di trovare limiti, d’altra parte richiede
spesso una programmazione molto professionale e si concentra esclusivamente
su una piattaforma impedendo un’agile riciclo dei propri sforzi su
altri mercati del mobile. Il non-nativo, invece, offre diversi vantaggi
ascrivibili alla necessità di creare applicazioni cross-platform distribuibili
su sistemi operativi diversi.\autocite{HTMLIT:PROGNATIVA} Non avendo
quindi necessità di mantenere alta la portabilità del codice su altre
piattaforme, si è preferito sviluppare tramite programmazione nativa;
potendo quindi sfruttare pienamente l'architettura del sistema operativo
sottostante e gli strumenti di sviluppo forniti ufficialmente.

Si è scelto quindi di utilizzare come principale strumento \emph{Android
Studio}, Integrated Development Environment (IDE) dedicato alla programmazione
Android distruibuito da JetBrains IntelliJ IDEA e Google che è ormai
il miglior software per la creazione dei applicazioni su questa piattaforma.
\autocite{ANDROIDDEVELOPERS:FIRSTAPP}

Durante lo sviluppo del progetto si è utilizzato come sistema di versioning
il software \textbf{Git}, così da poter mantenere uno storico del
lavoro svolto e ottenendo tutti i vantaggi del'utilizzo di un Sistema
per il Controllo di Versione (\textit{Version Control System - VCS}),
mentre per il test delle chiamate al backend della piattaforma si
è utilizzata la web application Postman \autocite{POSTMAN} che
permette di testare e sviluppare le APIs con cui due applicativi possono
dialogare.

\subsection{Programmazione Nativa}

Le applicazioni Android sono sviluppate utilizzando il linguaggio
Java. È un linguaggio di programmazione molto popolare sviluppato
da Sun Microsystems (ora di proprietà di Oracle) orientato agli oggetti
a tipizzazione statica, specificatamente progettato per essere il
più possibile indipendente dalla piattaforma di esecuzione. //https://it.wikipedia.org/wiki/Java\_(linguaggio\_di\_programmazione)

Per lo sviluppo di un'applicativo si devono sfruttare le principali
caratteristiche di Java andando ad estendere, per ogni schermata che
si vuole creare, la classe \emph{Activity }implementata come standard
all'interno delle libreria Android. Qualsiasi progetto sarà quindi
formato da un insieme di classi che rappresenteranno logicamente ogni
singola schermata unite ad altre classi che incapsuleranno invece
le funzioni, i modelli e ogni altro componente standard di Android
(Broadcast Receiver, Service, ecc...).All'interno dell'ultima distribuzione
di Android sono state incluse alcune delle funzionalità di Java 8,
ma l'alta frammentazione della piattaforma non ne permette correttamente
l'utilizzo. Questo si ripercuote su alcune funzionalità come le chiamate
http che non supportano alcune delle semplificazioni che sono state
fatte ad esempio nelle connessioni sicure grazie a TLS.

Come si può vedere all'interno dello snippet di codice la classe \emph{Main}
estende la classe \emph{AppCompatActivity }e dopo aver richiamato
il costruttore della propria superclasse esegue la creazione dell'internfaccia
utente grazie al file di layout \emph{activity\_main} che ne specifica
la struttura e le impostazioni, scritto in XML.

\inputencoding{latin9}\begin{lstlisting}
 public class Main extends AppCompatActivity {
    @Override
    protected void onCreate(Bundle savedInstanceState) {
        super.onCreate(savedInstanceState);
        setContentView(R.layout.activity_main);
    }
    ...
}
\end{lstlisting}
\inputencoding{utf8}
In informatica XML (sigla di eXtensible Markup Language) è un metalinguaggio
per la definizione di linguaggi di markup, ovvero un linguaggio marcatore
basato su un meccanismo sintattico che consente di definire e controllare
il significato degli elementi contenuti in un documento o in un testo.
All'interno dello sviluppo di applicazioni Android rappresenta il
linguaggio per la creazione delle risorse: layout, icone, manifest,
ecc...

L'esempio più ecclatante è l'utilizzo dell'XML per produrre il manifesto
dell'applicazione, che ne rappresenta ed espone la struttura principale
dichiarando ogni singola schermata o servizio presenti specificando
anche il tema da utilizzare, il nome e alcune opzioni di avvio. Sono
scritte in XML anche tutte le risorse interne all'applicazione come
si può notare nell'esempio di codice sottostante che descrive una
schermata con una singola immagine.

\inputencoding{latin9}\begin{lstlisting}
<?xml version="1.0" encoding="utf-8"?>
<LinearLayout
    xmlns:android="http://schemas.android.com/apk/res/android"
    android:layout_width="match_parent"
    android:layout_height="match_parent">
    <ImageView
         android:layout_width="wrap_content"
         android:layout_height="wrap_content"
         src="drawable/icon" />
</LinearLayout>
\end{lstlisting}
\inputencoding{utf8}

\subsection{Android Studio}

Android Studio \autocite{WIKIPEDIA:ANDROIDSTUDIO} è un ambiente
di sviluppo integrato (IDE) per lo sviluppo per la piattaforma Android.
È stato annunciato il 16 maggio 2013 in occasione della conferenza
Google I/O e la prima build stabile fu rilasciata nel dicembre del
2014. Basato sul software della JetBrains IntelliJ IDEA, Android Studio
è stato progettato specificamente per lo sviluppo di Android. Sostituisce
gli Android Development Tools (ADT) di Eclipse, diventando l' IDE
primario di Google per lo sviluppo nativo di applicazioni Android
offre ancora più funzioni che migliorano la produttività durante la
creazione di applicazioni Android, come ad esempio: un sistema di
compilazione Gradle, un ambiente unificato dove è possibile sviluppare
per tutti i dispositivi, esecuzione e costruzione di un file APK per
la pubblicazione di una applicazione sui canali di distribuzione più
importanti e l'integrazione GitHub.

Ogni progetto creato all'interno del software contiene alcuni moduli
principali che riguardano l'applicazione vera e propria, le librerie
incluse e i moduli proprietari di Google che permettono l'utilizzo
dei suoi servizi principali come Firebase e le Google Maps API.

A livello di build del sistema si trova uno script Gradle che a partire
da un manifesto, che dichiara la struttura dell'applicazione, i file
Java contenenti il codice sorgente (inclusi i test) e l'insieme delle
risorse e degli asset può generare un file apk firmato installabile
su qualsiasi device Android.

Per la generazione delle risorse Android Studio mette a disposizione
alcuni tool, tra cui alcuni grafici, che permettono l'implementazione
e il test dell'interfaccia utente senza dover di volta in volta avviare
l'applicativo su un emulatore, messo a disposizione dal sofwtare stesso,
o su un device fisico.

Ovviamente AS incapsula tutti i tool utili alla programmazione come
la formattazione, lo stile e la correzione in realtime del codice
scritto effettuando anche il controllo sintattico e un basilare confronto
semantico seguendo i pattern standard della piattaforma.

\subsection{Git}

Git è un software di controllo versione distribuito (\textit{Distributed
Version Control Systems - DVCS}) utilizzabile da interfaccia a riga
di comando, creato da Linus Torvalds nel 2005. \autocite{WIKIPEDIA:GIT}
Nacque per essere un semplice strumento per facilitare lo sviluppo
del kernel Linux ed è diventato uno degli strumenti di controllo versione
più diffusi.

Per quanto riguarda qualsiasi tipo di progetto di IT (\textit{Information
Technology}), specialmente se si tratta di un lavoro da dover svolgere
in team, Git da la possibilità di mantenere in memoria tutte le versioni
del proprio lavoro (sia in locale che online). Questo permette a più
persone di poter accedere alla cronologia del lavoro condiviso, avendo
anche la possibilità di verificare la presenza di errori e di eliminarli
tornando ad una versione precedente del progetto. Git sfrutta alcuni
algoritmi avanzati per calcolare le differenze riga per riga tra i
file di diverse versioni così da verificare la presenza di errori
o conflitti.

Git è fortemente direzionato verso uno sviluppo progettuale non lineare.
Supporta diramazione e fusione (\textit{branching and merging}) rapide
e comode, e comprende strumenti specifici per visualizzare e navigare
l'intero storico delle versioni anche se proposte da sistemi differenti.
È veloce e scalabile nella gestione di grandi progetti ed è molto
utile nello sviluppo in team, specialmente se affiancato dal modello
GitFlow che permette di gestire rami specifici per il developing o
per le release.

Infine grazie all'utilizzo di piccoli accorgimenti come il file .gitignore
(che permette di non salvare anche i file temporanei e non importanti
del progetto) e soluzioni online come GitHub o BitBucket (quest'ultimo
utilizzato durante lo sviluppo di questa tesi) il processo di sviluppo
software viene drasticamente avvantaggiato.

\section{Database}

La piattaforma Android fornisce diversi metodi e strumenti per salvare
i dati delle applicazioni in modo persistente. Per quanto riguarda
le preferenze relative alle impostazioni dell'applicazione si possono
implementare le \emph{Shared Preferences }per salvare le scelte dell’utente;
mentre per quanto riguarda la memorizzazione persistente di una grossa
mole di dati si possono utilizzare strumenti come l'\emph{Internal
Storage} o un \emph{Database SQLlite}. //cit http://www.html.it/articoli/la-gestione-dei-database-in-android-1/

Esistono però anche alternativa agli strumenti standard forniti ufficialmente
per la piattaforma e uno di questi è la libreria \emph{Realm}: database
object-oriented che non incapsula SQLite ma lo sostituisce con un
sistema di memorizzazione scritto in C++ che permette l'accesso ai
dati anche tramite l'utilizzo di linguaggi di programmazione differenti.
Modifica inoltre il metodo di accesso ai dati salvati non richiedendo
query esplicite ma semplici richieste in Java utilizzabili direttamente
all'interno del codice implementato, semplificando di gran lunga l'utilizzo
del database durante lo sviluppo di qualsiasi applicazione.

\subsection{Realm}

\section{Network Connection}

Attualmente ogni tipo di piattaforma che voglia permettere al proprio
utente di accedere da remoto, e quindi da qualsiasi dispositivo, alle
proprie informazioni ha come parte fondamentale lo sviluppo e il mantenimento
di un server di beckend (quindi nascosto nelle funzionalità alla vista
dell'utente) che mantenga, gestisca e restituisca i dati necessari
alle funzioni degli applicativi. Nello stesso modo la piattaforma
MyGelato si basa su un server che mantiene ogni informazione riguardante
utenti, shop e cards fornendo delle API REST che vengono sfruttate
dalle applicazioni mobile e web per poter fornire i propri servizi.

Il passaggio di informazioni tra i due applicativi avviene tramite
chiamate internet con protocollo Http protette grazie ad una connessione
criptata dal Transport Layer Security (TLS) o dal suo predecessore,
Secure Sockets Layer (SSL). Grazie alla API REST è possibile seguire
alcuni pattern delle tecnologie che permettono di valutare ogni possibile
errore delle connessioni: errori nei dati, nell connessione, ecc...
In Java le chiamate http sono difficilmente gestibili poichè non sono
automatizzate e son stati effettuati dei cambiamenti nelle ultime
versioni che non permettono di lavorare correttamente su tutti i dispositivi
Android. Per questo motivo è stata utilizzata la libreria open-source
OkHttp sviluppata da Square, che permette di automatizzare le chiamate
http sfruttando anche una notevole semplificazione del codice utilizzato
per programmare. \autocite{SQUARE:OKHTTP

\subsection{OkHttp}

La liberia OkHttp nasce dal fatto che il protocollo http è l'attuale
mezzo di comunicazione per ogni applicativo moderno. Per questo si
pone l'obiettivo di semplificarne l'utilizzo in più di una tecnologia
migliorandone le prestazioni e fornendo allo sviluppatore alcuni automatismi
comodi durante lo sviluppo e la programmazione.

Supporta ogni tipo di comunicazione http, ne gestisce la latenza,
il caching, la riconnessione, l'utilizzo di protocolli sicuri come
TLS, l'utilizzo di indirizzi IP multipli ed è molto semplice da integrare
e utilizzare all'interno del proprio software. Grazie ad un'interfaccia
molto semplice permette di automatizzare le chiamate http utilizzando
dei Builder che lasciano specificare solo i parametri necessari e
restituiscono un unico oggetto da cui si ottiene la risposta, positiva
o negativa, della richiesta.

\inputencoding{latin9}\begin{lstlisting}
OkHttpClient client = new OkHttpClient();
Request request = new Request.Builder()
           .url(endpoint)
           .addHeader("Accept","application/json")
           .build();

Response response = client.newCall(request).execute();
\end{lstlisting}
\inputencoding{utf8}
Inizializzato il client OkHttp serve creare un oggetto \emph{Request
}a cui vanno passati i parametri necessari per poi eseguire la chiamata
ottenendo infine un oggetto \emph{Response }all'interno del quale
sono presenti i dati scaricati, il codice di risposta della chiamata
e anche eventuali errori.

\section{Gestore di Eventi}

Un passaggio fondamentale durante lo sviluppo di un software è l'aggiornamento
delle view a seguito di un cambiamento nello stato dell'applicazione.
In Android è necessario intervenire direttamente sugli elementi dell'interfaccia
modificando ogni oggetto in base alle specifiche richieste senza poter
sfruttare veri e propri automatismi. Una forte limitazione del sistema
operativo è dovuta, per esempio, al fatto che l'unico thread che ha
il permesso di accedere e di modificare l'interfaccia utente è il
Main Thread; questo complica l'interazione con eventuali sistemi multithreading
anche se, ovviamente, fa parte delle specifiche di sistema utili a
limitare i conflitti sull'accesso alle risorse condivise (in questo
caso le view).

Questo problema necessita di ampia valutazione durante lo sviluppo
e anche in questo caso si è dovuto strutturare l'applicazione in modo
che non incorresse in errori di inconsistenza tra le view a cui il
programma voleva accedere e le view realmente visualizzate al momento.
Nel caso si utilizzino script asincroni è molto facile tentare di
accedere ad elementi, o anche intere schermate, non più presenti a
video; per questo motivo si è scelto di utilizzare la libreria EventBus
sviluppata da GreenRobot \autocite{GITHUB:EVENTBUS} e pensata esattamente
per gestire e risolvere questo tipo di problemi.

\subsection{EventBus}

EventBus è una libreria open-source per Android che utilizza il pattern
publish/subscribe per far dialogare tutti i componenti di un'applicazione.
Disaccoppia le classi che interagiscono sulla stessa interfaccia e
le gestisce in maniera centralizzata monitorando le performance e
gestendo gli errori che potrebbero essere sollevati da uno sviluppo
multithreading.\autocite{GREENROBOT:EVENTBUS}

I principali benefici che si riscontrano tramite l'utilizzo di questa
libreria sono la semplificazione delle comunicazioni tra componenti,
il disaccoppiamento tra mittenti e riceventi degli eventi, miglioramenti
nelle perfomance, facile integrazione dei metodi da utilizzare ed
infine anche caratteristiche avanzate come priorità e indirizzamento
a thread specifici.

Durante lo svolgimento di questa tesi si è definita una struttura
abbastanza comune formata da: - AsyncTask, oggetto che permette di
elaborare una serie di informazioni in background, utilizzato per
eseguire il download delle informazioni dal server; - View specifica
dell'activity con determinati componenti da aggiornare in base allo
stato dell'applicazione; - Metodo updateUI() che preso in gresso i
dati scaricati doveva aggiornare le view rispetto allo stato.

Per far dialogare questi elementi si è utilizzato EventBus così da
sfruttare il lavoro di thread in background senza limitare l'utente
nell'attesa di un determinato evento. Si procedeva quindi crendo un
evento specifico per l'interazione da dover gestire di cui qui viene
riportato un esempio:

\inputencoding{latin9}\begin{lstlisting}
public static class UpdateCouponUiEvent<T> {
	public T data;
	public UpdateCouponUiEvent() {
	}
	public UpdateCouponUiEvent(T data) {
		this.data = data;
	}
}
\end{lstlisting}
\inputencoding{utf8}
EventBus permette di registrare dei metodi eseguendo una subscribe
ad un determinato evento: ogni qual volta un evento di quel tipo viene
sollevato, il metodo viene eseguito. La subscribe deve essere definita
nel codice nel seguente modo:

\inputencoding{latin9}\begin{lstlisting}
@Subscribe
public void updateUI(UpdateCouponUiEvent event) {
	// update UI with event.data
}
\end{lstlisting}
\inputencoding{utf8}
Grazie all'utilizzo di eventi custom è dunque possibile passare anche
dei dati tramite l'evento che viene lanciato, nel codice messo ad
esempio si tratta della variable data, definito di tipo generico.

Infine all'interno del AsyncTask, a conclusione del download delle
informazioni e dell'aggiornamento dello stato, era sollevato un evento
grazie ad una publish:

\inputencoding{latin9}\begin{lstlisting}
EventBus.getDefault().post(new UpdateCouponUiEvent());
\end{lstlisting}
\inputencoding{utf8}

\section{Stripe}

\newpage{}
