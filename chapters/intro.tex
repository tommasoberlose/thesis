
\chapter*{Introduzione}

Questo momento storico è caratterizzato se non forse pienamente descritto
dall'esplosione dello sviluppo tecnologico che ha drasticamente cambiato
i metodi di comunicazione e il modo con cui le persone accedono alle
informazioni. Ed è proprio l'\emph{Informazione} che è diventata il
nodo fondamentale di ogni sistema economico moderno: tracciabilità,
trasparenza e accesso ai dati relativi al singolo prodotto sono solo
alcuni dei concetti che si sono diffusi grazie all'entrata di dispositivi,
come gli smartphone, nella quotidianità del pubblico consumer.\medskip{}

La larga diffusione di queste tecnologie standardizzate e disponibili
sul mercato con un target di acquirenti molto ampio, ha permesso all'utente
medio di poter accedere in ogni istante e in mobilità a internet,
realtà da cui può ottenere informazioni riguardanti le attività commerciali
a cui è maggiormente interessato e acquistare moltissimi prodotti
grazie alla diffusione di sistemi di reselling online.

\medskip{}

Tutte le fasi di commercializzazione di un prodotto o di un servizio,
sono quindi da rivalutare in considerazione di questo cambio di paradigma
che ha spostato online sia la pubblicizzazione che la vendita. Queste
nuove interazioni tra il produttore e il consumatore hanno però comportato
la nascita di un sistema di scambio di informazioni bidirezionale
molto più stretto: i sistemi di \emph{interest-based advertising}
permettono all'utente di avere consigli e pubblicità personalizzate
mentre le aziende ottengono la possibilità di creare sistemi di commercializzazione
efficaci e diversificati grazie alla presenza di moderne intelligenze
artificiali.\medskip{}

Vi è quindi stato un completo rinnovamento dei meccanismi alla base
del nostro sistema di mercato, che vede nel capitalismo la sua forma
più evoluta ma che proprio a causa della forte ingerenza della tecnologia
inizia a mostrare i primi segni di uno spostamento degli equilibri,
portando l'intero sistema a una deriva difficilmente prevedibile e
che si potrà assestare solo nel momento in cui le aziende prenderanno
consapevolezza di come molti settori stiano cambiando. Sarà sempre
più importante quindi tenere in considerazione l'inserimento di una
componente tecnologica in ogni ambito della produzione di un'azienda,
ma specialmente durante la fase di commercializzazione.\medskip{}

In questa tesi si tratterà la soluzione adottata dall'azienda Carpigiani,
leader mondiale nel settore della vendita di macchine per gelato artigianale,
nell'intraprendere lo sviluppo dell'ecosistema denominato \emph{MyGelato},
piattaforma di marketing digitale e di e-commerce fruibile tramite
applicazione mobile con l'obiettivo di dare nuovi strumenti ai propri
clienti. Si presenta il design e l’implementazione dell'applicazione
mobile legata a questo progetto e distribuita per il sistema operativo
Google Android. Le tecnologie scelte e la generalità della soluzione
consentono di impiegare gli stessi principi per altri progetti con
architetture analoghe.\medskip{}

La stesura di questo elaborato segue logicamente i passaggi fondamentali
che si presentano durante lo sviluppo di questo tipo di sistemi, partendo
dall'introduzione del contesto in cui si inserisce il lavoro in oggetto,
insieme alla valutazione delle finalità implicite ed esplicite che
hanno portato l'azienda Carpigiani a dare vita a questo ecosistema.

Nel primo capitolo si passa alla presentazione e alla valutazione
delle specifiche richieste ponendo maggiore attenzione ai sistemi
di marketing digitale ed e-commerce che rappresentano il fulcro dell'applicativo,
uniti ovviamente ai concetti principali che formano i modelli della
piattaforma e che devono essere considerati nel dettaglio per capire
alcune scelte strutturali effettuate durante lo sviluppo.

Sono poi introdotte le principali tecnologie, librerie e metodi di
sviluppo utilizzati durante l'implementazione di ogni singolo componente
dando ampio spazio alle funzionalità rese disponibili dall'utilizzo
di ottime librerie open source, descrivendo come hanno potuto semplificare
e velocizzare il processo lavorativo fornendo un forte imprinting
modulare all'architettura di tutta l'applicazione. \medskip{}

Si prosegue con la descrizione del lavoro svolto, discernendo ogni
singolo componente dell'applicativo così da poter spiegare nel dettaglio
le scelte implementative che hanno portato al prodotto finale di questa
tesi che, dopo una serie di test funzionali e prestazionali, potrà
essere pubblicato e reso disponibile sui principali mezzi di distribuzione
di applicazioni Android. Alcune considerazioni di carattere generale
e possibili sviluppi futuri concludono, infine, la tesi.

\newpage{}
