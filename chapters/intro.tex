%% LyX 2.2.2 created this file.  For more info, see http://www.lyx.org/.
%% Do not edit unless you really know what you are doing.
\documentclass[12pt,twoside,italian,openright,cleardoublepage=empty]{extreport}
\usepackage[T1]{fontenc}
\usepackage[utf8]{inputenc}
\usepackage[a4paper]{geometry}
\geometry{verbose}
\usepackage{fancyhdr}
\pagestyle{fancy}
\setcounter{secnumdepth}{3}
\setcounter{tocdepth}{3}
\usepackage{babel}
\usepackage[unicode=true,
 bookmarks=false,
 breaklinks=false,pdfborder={0 0 1},backref=section,colorlinks=false]
 {hyperref}
\hypersetup{
 colorlinks,citecolor=black,filecolor=black,linkcolor=black,urlcolor=black}
\begin{document}

\section*{Introduzione}

Questo momento storico è caratterizzato se non forse pienamente descritto
dall'esplosione dello sviluppo tecnologico che ha drasticamente cambiato
i metodi di comunicazione e il modo con cui le persone accedono alle
informazioni. Ed è proprio l'\emph{Informazione} che è diventata il
nodo fondamentale di ogni sistema economico moderno: tracciabilità,
trasparenza e accesso ai dati di un prodotto sia in termini di produzione
che di commercializzazione sono solo alcuni dei concetti che si sono
diffusi grazie all'entrata della tecnologia, come gli smartphone,
nella quotidianità del pubblico consumer.

Nel momento in cui l'utente accede a qualsiasi risorsa presente online,
si tratti anche solo della visualizzazione di un sito web, viene però
esposto ai nuovi sistemi di marketing digitale fondati sull'advertising
online che, uniti ai nuovi sistemi di e-commerce, sono diventati una
delle strumenti principali per la commercializzazione e vendita di
un prodotto.

Questi sono alcuni dei meccanismi alla base del nostro sistema di
mercato che vede nel capitalismo la sua forma più evoluta ma che proprio
a causa della forte ingerenza della tecnologia inizia a mostrare i
primi segni di uno spostamento degli equilibri, portando l'intero
sistema a una deriva difficilmente prevedibile e che si potrà assestare
solo nel momento in cui le aziende prenderanno consapevolezza di come
molti settori stanno cambiando. Sarà sempre più importante quindi
tenere in considerazione l'inserimento di una componente tecnologica
in ogni ambito della produzione di un prodotto, ma specialmente durante
la sua commercializzazione.

In questa tesi si tratterà la soluzione adottata dall'azienda Carpigiani,
leader mondiale nel settore della vendita di macchine per gelato,
nell'intraprendere lo sviluppo dell'ecosistema \emph{MyGelato}, piattaforma
di marketing digitale e di e-commerce fruibile tramite applicazione
mobile con l'obiettivo di dare nuovi strumenti ai propri clienti.
Si presenta il design e l’implementazione dell'applicazione mobile
legata a questo progetto e distribuita per il sistema operativo Google
Android. Le tecnologie scelte e la generalità della soluzione consentono
di impiegare gli stessi principi per altri progetti con architetture
analoghe.

La stesura di questo elaborato segue logicamente i passaggi fondamentali
che si presentano in progetti di questa tipologia partendo dall'introduzione
del contesto in cui si inserisce il lavoro in oggetto, insieme alla
valutazione delle finalità implicite ed esplicite che hanno portato
l'azienda Carpigiani a dare vita all'intero ecosistema MyGelato.

Nel capitolo 1 si passa alla presentazione e alla valutazione delle
specifiche richieste ponendo maggiore attenzione ai due sistemi di
marketing digitale ed e-commerce che rappresentano il fulcro dell'applicativo
uniti ovviamente ai concetti principali che formano i modelli della
piattaforma e che devono essere considerati nel dettaglio per capire
alcune scelte strutturali effettuate durante lo sviluppo.

Sono poi introdotte le principali tecnologie, librerie e i metodi
di sviluppo utilizzati durante l'implementazione di ogni singolo componente
dando ampio spazio alle funzionalità rese disponibili dall'utilizzo
di ottime librerie open source descrivendo come hanno potuto semplificare
e velocizzare il processo lavorativo fornendo un forte imprinting
modulare all'architettura di tutta l'applicazione. 

Si prosegue con la descrizione del lavoro svolto, discernendo ogni
singolo componente dell'applicativo così da poter spiegare nel dettaglio
le scelte implementative che hanno portato al prodotto finale di questa
tesi che dopo una serie di test in termini di prestazioni potrà essere
pubblicato e reso disponibile sui principali mezzi di distribuzione
di applicazioni Android.

Alcune considerazioni di carattere generale e possibili sviluppi futuri
concludono, infine, la tesi.

\newpage{}
\end{document}
