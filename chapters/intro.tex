
\chapter*{Introduzione}

Lo sviluppo tecnologico caratteristico del periodo a cavallo tra primo
e secondo millennio ha drasticamente cambiato i metodi di comunicazione
e il modo con cui le persone accedono alle informazioni. La gestione
di queste ultime è diventata il fulcro di ogni sistema economico moderno:
tracciabilità, trasparenza e accesso ai dati relativi al singolo prodotto
sono solo alcuni dei concetti che si sono diffusi, grazie all'entrata
nella quotidianità del pubblico consumer di dispositivi come gli smartphone.\medskip{}

Il processo di crescita esponenziale relativo alla diffusione di queste
tecnologie e la facilità con la quale esse sono reperibili sul mercato,
ha fatto sì che l'utente medio sia in grado di poter accedere quando
vuole in mobilità a internet, mezzo tramite il quale può informarsi
su ciò che lo interessa o, per esempio, acquistare moltissimi prodotti
grazie alla diffusione di sistemi di reselling online.

\medskip{}

Tutte le fasi di commercializzazione di un prodotto o di un servizio
sono quindi da rivalutare in considerazione della tendenza sempre
più frequente di spostarne online sia la pubblicizzazione che la vendita.
Le nuove interazioni tra il produttore e il consumatore venutesi a
creare hanno però comportato la nascita di un sistema di scambio di
informazioni bidirezionale molto più serrato: i sistemi di \emph{interest-based
advertising} permettono infatti all'utente di ricevere consigli e
avere pubblicità personalizzate; per le aziende esiste invece la possibilità
di creare sistemi di commercializzazione efficaci e diversificati
grazie alla presenza di moderne intelligenze artificiali.\medskip{}

Tutto ciò ha comportato un rinnovamento dei meccanismi alla base del
nostro sistema di mercato e, a causa della forte ingerenza della tecnologia,
si possono già vedere i segni di uno spostamento degli equilibri che
porteranno l’intero sistema commerciale a una deriva difficilmente
prevedibile che si potrà assestare solo nel momento in cui le aziende
prenderanno consapevolezza di tali cambiamenti. Sarà sempre più importante
quindi tenere in considerazione l’inserimento di una componente tecnologica
in ogni ambito della produzione di un’azienda ma specialmente durante
la fase di commercializzazione.

\medskip{}

All'interno di questa tesi si analizza la soluzione adottata dall'azienda
Carpigiani, leader mondiale nel settore della vendita di macchine
per gelato artigianale, per lo sviluppo dell'ecosistema denominato
\emph{MyGelato}, piattaforma di marketing digitale e di e-commerce
fruibile tramite applicazione mobile con l’obiettivo di dare nuovi
strumenti ai propri clienti.

\medskip{}

L’architettura dell'intero progetto si fonda su un backend Ruby on
Rails che implementa le funzionalità di creazione, gestione e fruizione
di contenuti multimediali affiancate da un sistema per la compravendita
di coupon digitali. Per poter usufruire di tali servizi è necessario
l’utilizzo di un’applicazione mobile disponibile per i sistemi operativi
mobile Apple iOS e Google Android: proprio quest'ultima è oggetto
di studio della presente tesi. Si è quindi presentata la progettazione
e l’implementazione descrivendo una soluzione che, grazie alla sua
generalità e alle tecnologie scelte, consente di impiegare gli stessi
principi per altri progetti con architetture analoghe.

\medskip{}

I concetti di marketing ed e-commerce rappresentano i flussi logici
principali che formano l’intero ecosistema MyGelato: il primo rispondendo
alla necessità da parte dei proprietari di gelaterie di trovare un
mezzo unificato e standardizzato per comunicare con i propri clienti,
il secondo proponendo al consumatore un sistema di buoni digitali
che, anche se già diffuso in altri ambiti, è fortemente innovativo
nel mercato della vendita di gelati artigianali.

\medskip{}

All'interno di questa tesi di laurea sono trattati tutti gli argomenti
contestualizzati per quanto riguarda lo sviluppo per applicazioni
mobile in ambito Android, presentando i pattern standard che si è
scelto di utilizzare e commentando le motivazioni che hanno portato,
in determinati casi, a scegliere strategie e metodologie differenti.
È stato deciso inoltre di porre un accento sulle modalità con cui
le tecnologie per smartphone siano diventate fondamentali per qualsiasi
progetto che si ponga tra le finalità di interagire con una gamma
di utilizzatori che vada dall'utente medio all'ambito business.

\medskip{}
La prima parte di questa tesi di laurea si occupa di descrivere il
contesto in cui si inserisce il lavoro in oggetto, insieme alla valutazione
delle finalità implicite ed esplicite che hanno portato l’azienda
Carpigiani a dare vita al progetto. Grande attenzione è posta sui
sistemi di marketing digitale ed e-commerce che rappresentano il fulcro
dell'applicativo, uniti ovviamente ai concetti principali che formano
i modelli della piattaforma e che devono essere considerati nel dettaglio
per capire alcune scelte strutturali effettuate durante lo sviluppo.

\medskip{}

Sono poi introdotte le principali tecnologie, librerie e metodi di
sviluppo utilizzati durante l’implementazione di ogni singolo componente
dando ampio spazio alle funzionalità rese disponibili dall'utilizzo
di ottime librerie open source, descrivendo come hanno potuto semplificare
e velocizzare il processo lavorativo fornendo un forte stampo modulare
all'architettura di tutta l’applicazione.

\medskip{}

L'esposizione del lavoro svolto continua discernendo ogni singolo
componente dell'applicativo così da poter spiegare nel dettaglio le
scelte implementative che hanno portato al prodotto finale di questa
tesi che, dopo una serie di test funzionali e prestazionali, potrà
essere pubblicato e reso disponibile sui principali mezzi di distribuzione
di applicazioni Android. Alcune considerazioni di carattere generale
sui possibili sviluppi futuri concludono la tesi.

\newpage{}
