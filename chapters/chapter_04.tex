\section{Implementazione}
A seguito della fase di progettazione si è scelto di suddivedere lo sviluppo dell'applicazione in alcune fasi principali che seguono logicamente i flussi presentati nel dettaglio all'interno del capitolo 2.
Questo ha permesso di dedicare maggiore attenzione ad ogni componente fino ad un alto livello di dettaglio; in modo da poter anche ottenere una buona valutazione sulle prestazioni dei nodi più critici dell'applicativo.

Per poter ottenere uno sviluppo abbastanza lineare è stato necessario considerare che alcune specifiche richieste permeavano tutto il sistema e che quindi non potevano essere sviluppate a se stante dagli altri componenti.
Prima tra tutto il design richiesto doveva essere presente in ogni singola view e per questo si è deciso di implementare fin da subito le interfaccie utenti per non dover replicare ad ogni passaggio le stesse operazioni di personalizzazioni della UI.

Strutturato il metodo di sviluppo per la grafica dell'applicazione si è passati a gestire la navigazione principale scegliendo di utilizzare alcuni dei principali pattern Android: il \textit{NavigationDrawer} e le \textit{RecyclerView}.
Il menù iniziale è servito a separare anche concettualmente le principali funzioni da dover sviluppare così da poter procedere successivamente con l'impletazione di ogni componente potendovi accedere anche se altre parti dell'applicazione non erano ancora disponibili.

Ragionando ulteriormente dal generale al dettaglio si è scelto di sviluppare la mappa di ricerca degli shop poichè presente in entrambi i flussi logici principali e quindi nodo cardine dell'applicazione, specialmente in termini di prestazioni.
La fase successiva ha fornito l'intera gestione della registrazione e dell'autenticazione di un utente poichè facente parte sia del sistema di marketing sia di quello di e-commerce, ultimi componenti implementati durante la fase di sviluppo.

A conclusione dell'implementazione si sono effettuati dei test per valutare le prestazioni dell'applicazione in modo particolare su quelle funzioni nodi centrali che avrebbero potuto inficiare l'esperienza utente se con basse prestazioni.


\subsection{Design}

\subsection{Ricerca}

\subsection{Utente}

\subsection{Marketing Digitale}

\subsubsection{Carte Promozionali}

\subsubsection{Preferiti}

Geofencing e notifiche push

\subsection{E-Commerce}

\subsubsection{Coupon}

\subsubsection{Acquisto}

\subsubsection{Condivisione e Riscatto}

\subsubsection{Utilizzo e Validazione}

\newpage